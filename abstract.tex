\documentclass[12pt,a4paper]{article}
\usepackage[utf8]{inputenc}
\usepackage[estonian]{babel}
\begin{document}
\title{Alglaadur ESTCube-1 k"asu- ja andmehalduss"usteemile ja kaameramoodulile}
\author{Karl Tarbe}
\maketitle

\begin{abstract}
T"o"o eesm"argiks on disainida ja implementeerida tudengisatelliidi
ESTCube-1 kahele alammoodulile alglaadur, mis v~oimaldaks nende
moodulite tarkvara uuendada ka siis, kui tehiskaaslane on juba orbiidil.
K"asu- ja andmehalduss"usteemil ja kaameramoodulil on sarnased
mikrokontrollerid --- vastavalt STM32F1 ja STM32F2 seeriasse kuuluvad,
mist~ottu saab m~olemal moodulil kasutada "uldjoontes sama
alglaadurit. 

Maa pealt uue tarkvara vastuv~otmine ja sobivasse kohta paigutamine
kuulub p~ohitarkavara "ulesannete hulka. Alglaaduri t"o"oks on lugeda
v"alisest m"alust k"askude nimekiri, k"asud t"aita ning
seej"arel k"aima panna p~ohitarkvara. 

Toetatud k"aske on v"ahemalt kaks. Esimene on v"alisest m"alust uue tarkvara
kopeerimine sisemises m"alus olevasse pessa. Teine v~oimaldab valida,
milline sisemises m"alus olevatest tarkvaradest k"aima panna. 

Alglaadur peab enne iga operatsiooni t"aitmist aruvutama tarkvara
kontrollsumma ning seda kontrollima, et v"altida vigase tarkvara
kopeerimist v~oi k"aivitamist.

Samuti peab alglaadur sisemisse m"alusse k~oik oma
tegevused kirja panema, et p~ohitarkvara saaks hiljem teada alglaaduri
t"o"os esinenud vigadest ning saaks andmed vigade kohta tagasi maale
saata. Sisemine m"alu on valitud just sellep"arast, et saaks kirja panna
vead ka siis, kui suhtlemine v"alimise m"aluga eba~onnestub.

Sisemiseks m"aluks on mikrokontrolleri sees olev FLASH t"u"upi m"alu.
Kasutusel olevad mikrokontrollerid suudavad programmi k"aivitada ainult
sisemisest FLASH m"alust v~oi sisemisest SRAM m"alust. Seet~ottu ongi vajalik, et
alglaadur paigutaks tarkvara sisemisse FLASH m"allu, sest SRAM t"u"upi
m"alu peab j"a"ama vabaks tarkvara enda t"o"oks. V"alimise m"aluna
on kasutusel I2C siinil olev mikrokontrollerist v"aljaj"a"av
ferroelektriline RAM t"u"upi m"alu, mis on kosmoses oleva
radiatsiooni vastu kordi kindlam, kui teised m"alut"u"ubid.

Samuti v~oib "ara m"arkida, et kasutatavates mikrokontrollerites on sisseehitatud
moodul kontrollsumma arvutamiseks, mida saab kasutada, et kontrollsumma
arvutamine kiiremini l"aheks.
\end{abstract}
\end{document}
