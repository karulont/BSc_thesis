\documentclass[12pt,a4paper]{article}
\usepackage[utf8]{inputenc}
\usepackage[estonian]{babel}
\usepackage{cite}

\title{Alglaadur ESTCube-1 käsu- ja andmehaldussüsteemile ja kaameramoodulile}
\author{Karl Tarbe}

\begin{document}

\begin{titlepage}
\begin{center}
Tartu Ülikool\\
Matemaatika-informaatika teaduskond\\
Arvutiteaduse intituut\\
Informaatika eriala
\vfill
Karl Tarbe\\[1cm]
Alglaadur ESTCube-1 käsu- ja andmehaldussüsteemile ja kaameramoodulile\\[4mm]
Bakalaureusetöö (6 EAP)
\vspace{2cm}
\begin{flushright}
	Juhendajad: Meelis Roos\\
	Helle Hein
\end{flushright}
\vfill
Tartu 2013
\end{center}
\end{titlepage}

\tableofcontents

\section{Sissejuhatus}
2todo todo todo todo todtodo todo todo o todo todo todo todo todo todo todo todo todo todo
2todo 2todotodo todo todo todo todo todo todo
t!!!odo t2todo todo todo todo todo todo todo todo todo todo todo todo todo todo
2todotodo tod 

\section{Süsteemi nõuded}
\section{Alglaadimine}
\subsection{Tavapärane alglaadimisprotseduur}
Kui mikrokontrolleril toide olemas on üritab see ennast tööle panna.
Esmalt lülitatakse sisse sisemine 8-MHz-ne ossillaator, mille signaali
neljandal tõusval frondil loetakse konfigureerimise viikude väärtused. Nende
väärtuste põhjal otsustatakse, millisest kohast alglaadida. Võimalikud
variandid on: põhi välkmälu, süsteemi mälu ja staatiline juhupöördlusega
mälu(SRAM).  Enim kasutatakse kahte esimest: põhi välkmälu seal oleva programmi
käivitamiseks ja süsteemi mälu, et põhimälu muuta. Tehases paigaldatakse
süsteemi mällu, mis on tegelikult lihtsalt üks kaitstud osa põhi välkmälu
lõpust, alglaadur, mille abil saab programmeerida ülejäänud välkmälu ehk oma
programmi mikrokontrollerile laadida.

Oletades, et programm on juba kiibi välkmälus olemas ja alglaadimist
alustatakse põhi välkmälust, vaatleme edasist protsessi. Välkmälu aadress on
tegelikult 0x800 0000, aga viikude oleku tõttu peegeldatakse see ka aadressile
0x0000 0000, kust protsessor oma tööd alustab.  Protsessor võtab aadressil
0x0000 0000 oleva väärtuse ja seab selle pinu ülemiseks väärtuseks. Seejärel
alustatakse instruktsioonide täitmist lähtestamise vektorist, mille
algusaadress on kirjas aadressil 0x0000 0004. \cite{f1rm}

Lähtestamise vektoril on tavaliselt assembleris kirjutatud protseduur, mis
kopeerib programmi andmete sektsiooni välkmälust SRAMi, täidab ülejäänud
programmi jaoks vajaliku ala SRAMist nullidega, kutsub välja C koodis kirjutatud
protseduuri, mis initsialiseerib süsteemi kella, ja siis annab juhtimise üle C
koodis kirjutatud \textit{main} protseduurile. Satelliidi peal ei kasutata
üldist kella seadmise koodi ja seega see samm jäetakse vahele ning kella
seadmine kutsutakse välja \textit{main}ist.

See, mida tarkvara edasi initisaliseerib, on juba tarkvara spetsiifiline.
Tõenäoliselt initsialiseeritakse mõned riistvaralised moodulid näiteks USART või
SPI ja hakatakse siis rakendusele vajalikku kontrolltsüklit täitma. Satelliidil
Estcube-1 kasutatakse nii CAMil kui ka CDHSil reaalaja operatsioonisüsteemi
FreeRTOS.

\subsection{Programmi alglaadimine koos alglaaduriga}
Alglaadur ise käitub alglaadimise seisukohalt nagu iga teinegi programm ja
laetakse täpselt samamoodi nagu eelmises alapeatükis kirjeldatud. See tähendab
seda, et alglaaduri katkestusvektorite tabel peab asuma välkmälu alguses, sest
sealt hakkab protsessor seda otsima. Kuna üks osa alglaadurist asub juba
välkmälu alguses, on ka teised osad sinna järele paigutatud, mis omakorda
tähendab, et tegelik põhiprogramm peab nüüd asuma kusagil mujal kui välkmälu
alguses ja sellega peab arvestama põhiprogrammi linkimisel.

Kui alglaadur on oma muud ülesanded juba täitnud, on tema viimaseks tööks
põhitarkvara käivitamine. Selleks käitub sarnaselt protsessoriga, kuid
kõigepealt on tal vaja teada, mis aadressil põhitarkvara katkestusvektorite
tabel asub. See aadress on aga eelnevalt kokkulepitud ja sissekodeeritud.

Enne põhitarkvara töölepanemist tuleks puhtaks teha pinu. Alglaaduri enda töö
ajal sinna salvestatud väärtusi pole põhiprogrammil kohe kuidagi vaja. Pinu
puhtaks tegemine tähendab lihtsalt pinu viida algväärtustamist, sest nii
hakatakse edaspidi pinus juba olevaid andmeid üle kirjutama ja kui programm on
korrektne, siis midagi sellist ei loeta, mida programm ise pole juba üle
kirjutanud. Algväärtustamiseks kasutatakse põhitarkvara katkestusvektorite
tabeli esimest elementi, mis näitas pinu ülemist aadressi.

Kuigi mikrokontrolleri käivitamise hetkel alustatakse tööd kindlast kohast, ei
pea terve katkestusvektorite tabel välkmälu alguses olema. Katkestustega tegeleb
eraldi riistvara moodul NVIC (\textit{Nested Vectored Interrupt Controller}).
Lisaks on sellega tihedalt seotud teine moodul SCB (\textit{System Control
Block}), kus asub seadistatav register VTOR (\textit{Vector Table Offset
Register}), mille väärtust kasutatakse katkestusvektorite tabeli aadressina.
Vaikimisi on seal väärtus 0x0000 0000, mida kasutatakse alglaaduri laadimiseks,
kuid alglaadur seab selleks väärtuseks põhiprogrammi katkestusvektorite tabeli
aadressi. Kui seda ei seadistataks, siis kutsutaks põhiprogrammi katkestuse ajal välja
alglaaduri katkestusvektoreid, mis on tegelikult implementeerimata, sest
alglaadur ei kasuta katkestusi, ja nii ei saaks põhitarkvara oma ülesandeid
täita. Lisaks sellele on vaikimisi katkestuse teenindamise rutiin lihtsalt
lõputu tsükkel, seega jääks kontroller katkestusse kinni ja halvatud ei oleks
ainult põhitarkvara katkestuse teenindamine vaid kogu põhitarkvara töö peatuks.

Kui pinu viit ja katkestusvektorite tabeli asukoht on ära seadistatud jääb üle
veel lugeda katkestusvektorite tabelist põhiprogrammi lähtestamise vektori
aadress ja mikrokontrolleri töö sinna edasi juhtida. Selleks tehakse C-koodis
funktsiooni viit, mis väärtustatakse lähtestamise vektori aadressiga ning
seejärel kutsutakse see välja.


\section{Komponendid}
\section{Kokkuvõte}

\bibliographystyle{plain}
\bibliography{viited}

\end{document}
