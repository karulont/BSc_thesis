\documentclass[12pt,a4paper]{article}
\usepackage[utf8]{inputenc}
\usepackage[estonian]{babel}
\usepackage{cite}

\title{Alglaadur ESTCube-1 käsu- ja andmehaldussüsteemile ja kaameramoodulile}
\author{Karl Tarbe}

\begin{document}

\begin{titlepage}
\begin{center}
Tartu Ülikool\\
Matemaatika-informaatika teaduskond\\
Arvutiteaduse intituut\\
Informaatika eriala
\vfill
Karl Tarbe\\[1cm]
Alglaadur ESTCube-1 käsu- ja andmehaldussüsteemile ja kaameramoodulile\\[4mm]
Bakalaureusetöö (6 EAP)
\vspace{2cm}
\begin{flushright}
	Juhendajad: Meelis Roos\\
	Helle Hein
\end{flushright}
\vfill
Tartu 2013
\end{center}
\end{titlepage}

\tableofcontents

\section{Sissejuhatus}
plaasadapldaspdlapdlapsd

\section{Süsteemi nõuded}
\section{Alglaadimine}
\subsection{Tavapärane alglaadimisprotseduur}
Kui mikrokontrolleril toide olemas on üritab see ennast tööle panna.
Esmalt lülitatakse sisse sisemine 8-MHz-ne ossillaator, mille signaali
neljandal tõusval frondil loetakse konfigureerimise viikude väärtused. Nende
väärtuste põhjal otsustatakse, millisest kohast alglaadida. Võimalikud
variandid on: põhi välkmälu, süsteemi mälu ja staatiline juhupöördlusega
mälu(SRAM).  Enim kasutatakse kahte esimest: põhi välkmälu seal oleva programmi
käivitamiseks ja süsteemi mälu, et põhimälu muuta. Tehases paigaldatakse
süsteemi mällu, mis on tegelikult lihtsalt üks kaitstud osa põhi välkmälu
lõpust, alglaadur, mille abil saab programmeerida ülejäänud välkmälu ehk oma
programmi mikrokontrollerile laadida.

Oletades, et programm on juba kiibi välkmälus olemas ja alglaadimist
alustatakse põhi välkmälust, vaatleme edasist protsessi. Välkmälu aadress on
tegelikult 0x800 0000, aga viikude oleku tõttu peegeldatakse see ka aadressile
0x0000 0000, kust protsessor oma tööd alustab.  Protsessor võtab aadressil
0x0000 0000 oleva väärtuse ja seab selle pinu ülemiseks väärtuseks. Seejärel
alustatakse instruktsioonide täitmist lähtestamise vektorist, mille
algusaadress on kirjas aadressil 0x0000 0004. \cite{f1rm}

Lähtestamise vektoril on tavaliselt assembleris kirjutatud protseduur, mis
kopeerib programmi andmete sektsiooni välkmälust SRAMi, täidab ülejäänud programmi jaoks
vajaliku ala SRAMist nullidega, kutsub välja C koodis kirjutatud
protseduuri, mis initsialiseerib süsteemi kella, ja siis annab juhtimise üle C
koodis kirjutatud \textit{main} protseduurile.

\section{Komponendid}
\section{Kokkuvõte}

\bibliographystyle{plain}
\bibliography{viited}

\end{document}
